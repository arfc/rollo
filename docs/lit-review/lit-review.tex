\documentclass[letterpaper,11pt]{article}
\usepackage[top=1.0in,bottom=1.0in,left=1.0in,right=1.0in]{geometry}
\usepackage{verbatim}
\usepackage{amssymb}
\usepackage{graphicx}
\usepackage{longtable}
\usepackage{amsfonts}
\usepackage{amsmath}
\usepackage{hyperref}
\usepackage{float}
\usepackage{setspace}
\def\thesection       {\arabic{section}}
\def\thesubsection     {\thesection.\alph{subsection}}

\author{Gwendolyn J. Chee}

\title{Generative Reactor Design}
\begin{document}
\maketitle
\hrulefill
\onehalfspacing

\section{Generative Design}
% What is generative design? How does it work? 
Generative design is an exploratory design method that autonomously 
generates optimal designs by iteratively varying design geometry 
to meet designer-defined performance metrics \cite{krish_practical_2011,oh_deep_2019}. 
Generative design varies the parameters of the problem definition
\cite{matejka_dream_2018}. 
At each iteration step, the design is evaluated on the 
performance metrics. 
Based on the results, the generative design algorithm changes the 
interval allowed for each design geometry variable, refining 
design constraints (problem definition) and moving towards 
designs that best meets performance metrics. 
Generative design is used in the following industries: 
automotive, space, aerospace, etc. 
In these industries, the goal of generative design is to decrease 
the weight of the vehicle while ensuring each part can continue to 
withstand the stresses and strains that are put on it within a 
safety factor. 
These industries can rely on commerical softwares such as 
Autodesk Fusion360 \cite{autodesk_autodesk_2020}.
to produce their generative designs, since these softwares have 
capabilities to evaluate the physics they require such as stress 
and strain. 
Whereas, for nuclear reactors, generative design is constrained by 
variables such as mass or volume of fuel, fuel enrichment, effectiveness 
of heat transfer, effective neutron multiplication factor, etc. 
Nuclear reactors not only experience physical forces, they also
require evaluation of the neutronics, therefore generative design of 
a nuclear reactor cannot make use of tools such as Autodesk Fusion360. 
Therefore, a framework that couples well-developed machine 
learning tools with well-supported monte-carlo particle transport 
codes (Serpent, MCNP, etc.) and thermal hydraulics software (RELAP7 etc.)
must be created to successfully produce generative reactor designs. 



\bibliographystyle{plain}
\bibliography{2020-generative-reactor-design-lit-review}
\end{document}

